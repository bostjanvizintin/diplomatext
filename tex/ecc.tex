\documentclass[12pt,a4paper,titlepage,openany]{report}
\usepackage{zakljucna_FAMNIT_1_stopnja_MA_MEF_2016}

% Glava dokumenta:

\fancyhf{}
\lhead[]{{\fontsize{9.3}{12}\selectfont
Priimek I. Naslov zaključne naloge.\\
\noindent Univerza na Primorskem, Fakulteta za matematiko, naravoslovje in informacijske tehnologije, leto}}
\chead[]{\fancyplain{}{}}
\rhead[]{\fancyplain{\thepage}
{\thepage}}
\cfoot[]{\fancyplain{}{}}
\lfoot[]{\fancyplain{}{}}
\rfoot[]{\fancyplain{}{}}
\normalsize

%%%%%%%%%%%%%%%%%%%%%%%%% ZAČETEK DOKUMENTA %%%%%%%%%%%%%%%%%%%%%%%%%%%%%%%%%%%%%%%%%%5

%%%%%%%%%%%%%%%%%%%%%%%%% Naslovna stran %%%%%%%%%%%%%%%%%%%%%%%%%


\begin{document}
\pagenumbering{Roman}
\pagestyle{empty}
\begin{center}
\noindent \large UNIVERZA NA PRIMORSKEM\\
\large FAKULTETA ZA MATEMATIKO, NARAVOSLOVJE IN\\
INFORMACIJSKE TEHNOLOGIJE


\normalsize
\vspace{6cm}
Zaključna naloga\\
\textbf{\large Nadzor porabe električne energije z namensko razivtim merilcem in  spletnim portalom}\\
\normalsize
(Control of electrical usage with dedicated measurer and  web portal)\\
\end{center}

\begin{flushleft}
\vspace{5cm}
\noindent Boštjan Vižintin:
% v zgornjo vrstico dopišite ime in priimek študenta
\\
\noindent Računalništvo in informatika:
% v zgornjo vrstico dopišite ime študijskega programa
\\
\noindent Tatjana Zrimec:
% v zgornjo vrstico dopišite akademski naziv, ime in priimek mentorja
\\
%\noindent Somentor:
% če imate somentorja, v zgornjo vrstico dopišite akademski naziv, ime in priimek somentorja
% če somentorja nimate, zbrišite zgornjo in spodnjo vrstico
%\\
\end{flushleft}

\vspace{4cm}
\begin{center}
\large \textbf{Koper, Avgust 2016}
% dopišite mesec in leto oddaje zaključne naloge
\end{center}
\newpage

\pagestyle{fancy}
%%%%%%%%%%%%%%%%%%%%%%%%%%%%%%% Ključna dokumentacijska informacija (slo in ang) %%%%%%%%%%%

\section*{Ključna dokumentacijska informacija}

\medskip
\begin{center}
\fbox{\parbox{\linewidth}{
\vspace{0.2cm}
\noindent
Ime in PRIIMEK:\vspace{0.5cm}\\
Naslov zaključne naloge:\vspace{0.5cm}\\
Kraj:\vspace{0.5cm}\\
Leto:\vspace{0.5cm}\\
Število listov: \hspace{2cm} Število slik: \hspace{2.6cm} Število tabel:\hspace{2cm}\vspace{0.5cm}\\
Število prilog: \hspace{1.9cm} Število strani prilog: \hspace{1cm} Število referenc:\vspace{0.5cm}\\
Mentor:\vspace{0.5cm}\\
Somentor:\vspace{0.5cm}\\
Ključne besede:\vspace{0.5cm}\\
Math.~Subj.~Class.~(2010):\vspace{0.5cm}\\
{\bf Izvleček:}\\
Izvleček predstavlja kratek, a jedrnat prikaz vsebine naloge. V največ 250 besedah nakažemo problem, metode, rezultate, ključne ugotovitve in njihov pomen.
\vspace{0.2cm}
}}
\end{center}

\newpage

\section*{Key words documentation}

\medskip

\begin{center}
\fbox{\parbox{\linewidth}{
\vspace{0.2cm}
\noindent
Name and SURNAME:\vspace{0.5cm}\\
Title of final project paper:\vspace{0.5cm}\\
Place:\vspace{0.5cm}\\
Year:\vspace{0.5cm}\\
Number of pages:\hspace{1.6cm} Number of figures:\hspace{2.2cm} Number of tables:\vspace{0.5cm}\\
Number of appendices:\hspace{0.6cm} Number of appendix pages:\hspace{0.8cm}Number of references:\vspace{0.5cm}\\
Mentor: title~First Name~Last Name, PhD\vspace{0.5cm}\\
% opomba: za "title" vpišite eno od naslednjega:
% Assist.~Prof. (če je naziv docent),
% Assoc.~Prof. (če je naziv izredni profesor),
% Prof. (če je naziv profesor)
Co-Mentor:\vspace{0.5cm}\\
Keywords:\vspace{0.5cm}\\
Math.~Subj.~Class.~(2010):\vspace{0.5cm}\\
{\bf Abstract:}
\vspace{0.2cm}
}}
\end{center}




%%%%%%%%%%%%%%%%%%%%%%%%%%%%%%% Zahvala %%%%%%%%%%%%%%%%%%%%%%%%%%%%%%%%%%%%%

\newpage
\section*{Zahvala}


Tu se zahvalimo sodelujočim pri zaključni nalogi, osebam ali ustanovam, ki so nam pri delu pomagale ali so delo omogočile. Zahvalimo se lahko tudi mentorju in morebitnemu somentorju.

%%%%%%%%%%%%%%%%%%%%%%%%%%%%% Kazala %%%%%%%%%%%%%%%%%%%%%%%%%%%%%%
\newpage

% Dodamo kazala (po potrebi):
\tableofcontents
\addtocontents{toc}{\protect\thispagestyle{fancy}}
\newpage
\listoftables
\addtocontents{lot}{\protect\thispagestyle{fancy}}
\newpage
\listoffigures
\addtocontents{lof}{\protect\thispagestyle{fancy}}
\newpage
% ker priloge niso oštevilčene, tudi pikic do številk strani (ki jih ni) ne izpišemo
\renewcommand{\cftdot}{}
\listofappendices
\thispagestyle{fancy}
\newpage

\chapter*{Seznam kratic}
\thispagestyle{fancyplain}
\begin{longtable}{@{}p{1cm}@{}p{\dimexpr\textwidth-1cm\relax}@{}}
\nomenclature{$tj.$}{to je}
\nomenclature{$npr.$}{na primer}
\end{longtable}
\newpage

\normalsize

%%%%%%%%%%%%%%%%%%%%%%%%%%%%%%%%%% Poglavja: %%%%%%%%%%%%%%%%%%%%%%%%%%%%%%%%%%%%%

% Namig: Za večjo preglednost datoteke lahko vsebino vsakega poglavja shranite v poseben .tex dokument
% v isto mapo, kjer je shranjena osnovna .tex datoteka. Nato poglavja vstavite v dokument s klicem \include
% Primer: PrvoPoglavje.tex in DrugoPoglavje.tex vstavimo tako:
% \include{PrvoPoglavje}
% \include{DrugoPoglavje}

\chapter{Uvod}
\thispagestyle{fancy}
\pagenumbering{arabic}

Tu opišemo problem, ki ga v zaključni nalogi obravnavamo. Predstavimo osnovne ideje in uvedemo
osnovne definicije in oznake. V uvodu lahko tudi povzamemo matematična dejstva, ki jih bomo
kasneje uporabili. Citiramo literaturo, ki je relevantna za obravnavane pojme, lahko tudi dodatno literaturo.

\chapter{Definicija problema}
\thispagestyle{fancy}
\pagenumbering{arabic}

Poraba električne energije v gospodinjstvih narašča in je v letu 2010 znašala približno 3.5 kWh na povprečno gospodinjstvo. Narašča tudi delež gospodinjstev opremljenih z dobrinami, ki za svoje delovanje potrebujejo elektriko. Na primer pomivalni stroj, stroj za sušenje perila, mobilni telefon, CD naprave, mikrovalovna pečica ter osebni računalnik. Kljub izboljšanju energetske učinkovitosti nekaterih naprav se poraba elektrike v povprečju ne znižuje, saj število naprav v gospodinjstvih narašča. Ker gospodinjstva porabijo tretjino skupne električne energije(http://shrinkthatfootprint.com/how-do-we-use-electricity), bi lahko z majhno spremembo porabe v posameznih gpospodinjstvih veliko vplivali na skupno porabo električne energije. Gospodinjstva redno prejemajo račune za količino porabljene električne energije vendar uporabniku skupna poraba električne energije ne koristi pri predstavi, kje v gospodijstvu bi lahko porabo omejil. Ideja je torej načrtovati in izdelati tak sistem, ki bi uporabnikom omogočal natančnejši nadzor porabe električne energije in s tem prikazal možnosti zmanjšanja električne energije.

\chapter{Rešitev problema}
\thispagestyle{fancy}
\pagenumbering{arabic}

Porabo električne energije v gospodinjstvih lahko zmanjšamo, vendar je to veliko enostavneje, če imamo natančen vpogled v porabo posameznih električnih naprav, ki si jih lasti gospodinjstvo. Glavna funkcija sistema bi torej bila nadzor porabe električne energije. Uporabniki bi lahko sami določili merilne točke in tako natačno izmerili porabo električne energije na posameznih merilnih mestih. Točna poraba bi bila nato prikazana za posamezna merilna mesta v obliki grafa. Poleg nadzora porabe, bi si lahko uporabniki določili različna opozorila. Sistem bi jih glede na opozorila opozarjal na preveliko uporabo oziroma na prekoračitev porabe električne energije in imel možnosti izklopa porabnikov v primeru sprožitve opozorila. Za lažjo predstavo o količini porabljene električne energije bi sistem omogočal tudi primerjavo porabe električne energije med seboj podobnimi gospodinjstvi.

\chapter{Težave}
\thispagestyle{fancy}
\pagenumbering{arabic}

Ena od težav, ki nastopi pri merjenju porabe električne energije je montaža merilcev. Ker povprečno gospodinjstvo ne obvlada elektronike, bi morali biti merilci enostavni za montažo, sicer bi vsako montažo moral opraviti strokovnjak kar bi povečalo stroške in tako odvrnilo zanimanje s strani uporabnikov. Težave nastopijo tudi zaradi velike količine podatkov, ki jih merilne naprave konstantno pošiljajo na strežnik. 

\chapter{Ekonomska izvedljivost}
\thispagestyle{fancy}
\pagenumbering{arabic}

Ena od omenjenih težav je velika količina podatkov ki se konstantno pošilja na strežnik, kar pomeni da bi strežnik moral biti zelo zanesljiv, saj igra ključno vlogo v sistemu. Taki strežniki lahko predstavljajo velik strošek, zato je nujno potreben nekakšen prihodek s strani uporabnikov sistema. Sistem bi zato omogočal prosto verzijo in plačljivo verzijo, katera nebi imela nekaterih omejitev ki bi jih imela prosta verzija. 
Poleg strežnika je potrebno načrtovanje in izdelava merilcev, za kar bi potrebovali nekaj začetnega kapitala. 
Pri izdelavi prototipa merilca in spletne aplikacije za diplomsko nalogo so stroški zanemarljivi, saj bomo za izdelavo merilca uporabili razvijalno ploščico arduino, ter njemu kompatibilne senzorje, za strežnik pa bo zadostoval navaden osebni računalnik.

\chapter{Analiza in definiranje zahtev}
\thispagestyle{fancy}
\pagenumbering{arabic}

Sistem za nadzor porabe električne energije bo na voljo vsem uporabnikom, ki bi želeli natančnejši pogled svoje porabe električne energije in tako potencialno zmanjšati porabo, kjer je to le možno. Sistem bo podpiral dve vrsti uporabnikov:
\begin{itemize}
\item Proste uporabnike
\item Plačljive uporabnike
\end{itemize}
Funkcije sistema bodo enake za vse uporabnike z razliko, da bodo imeli prosti uporabniki nekaj omejitev pri nekaterih funkcijah kot na primer:
\begin{itemize}
\item Omejeno število meritev posameznega merilnega mesta na dan (manj natančna poraba)
\item Omejeno število dni hranjenja starih meritev
\item Omejeno število merilnih mest
\item Omejeno število nastavljivih opozoril
\end{itemize}


\section{Funkcijske zahteve}
\thispagestyle{fancy}
\pagenumbering{arabic}

\subsection{Funkcijske zahteve sistema:}
\begin{itemize}

\item Registracija uporabnika v sistem
\item Prijava uporabnika v sistem
\item Nadgradnja uporabnika v plačljivega uporabnika
\item Vnos novega merilca
\item Urejanje vnešenih merilcev
\item Imenovanje posameznih merilnih mest že vnešenih merilcev
\item Urejanje že vnesenih merilnih mest
\item Dodajanje novih opozoril
\item Urejanje že vnesenih opozoril
\item Pregled porabe električne energije po merilnih mestih z možnostjo različnih filtrov
\item Pogled porabe posameznega senzorja "v živo"
\item Pregled napak senzorja/sistema
\item Pregled navodil za uporabo
\item Nastavitev stikal za vklop/izklop merilnih mest/porabnikov

\end{itemize}

\subsection{Prijava/registracija v sistem:}
Kot večina aplikacij, bo sistem omogočal uporabnikom registracijo in nato prijavo v sistem. Poleg prijave in registracije bodo uporabniki imeli možnost nadgradnje svojega računa in si s tem omgočili dodatne funkcije sistema, ki ne bodo mogoče navadnim uporabnikom.

\subsection{Vnos in urejanje novega merilca:}
Uporabniki bodo za merjenje potrebovali posebne merilce, razvite točno za ta namen, katere bo mogoče kupiti. Merilci bojo med seboj ločeni s posebnim nizem znakov, ki bo vnaprej določen in ga bo moral uporabnik vnesti v sistem ob vnosu novega merilca. Merilcu bo uporabnik ob vnosu v sistem določil še ime in število merilnih mest. Že vnesenim merilcem bo lahko uporabnik spremenil ime in število merilnih mest.

\subsection{Imenovanje in urejanje merilnih mest:}
Uporabiki bodo morali pred začetkom opravljanja meritev, imenovati merilne točke. Vsaka točka bo imela svoje ime in pod ime. Merilna mesta, ki bodo merila na primer vsako vtičnico posebej bodo imela enako ime ampak različno pod ime. Imena bo mogoče tudi spremeniti.

\subsection{Dodajanje in urejanje opozoril:}
Uporabniki bodo lahko izbirali med dvema vrstama opozoril:
\begin{itemize}
\item Ko bo trenutna poraba večja od določene vrednosti, ki jo bo določil uporabnik
\item Ko bo skupna poraba na merilnem mestu presegla vrednost, ki jo bo določil uporabnik.
\end{itemize}
Uporabnik bo določil ime, tip, obdobje in vrednost opozorila. Ko bo sistem zaznal da je neka vrednost presegla vrednost opozorila, bo o tem obvestil uporabnika (v obliki e-mail sporočila). Uporabniku bo ob prihodu na stran z opozorili prikazano katera opozorila so presegla vrednost.
Ko se bodo opozorila sprožila bodo postala neaktivna. Uporabnik bo lahko ta opozorila, vključno z ostalimi urejal (spreminjal imena, vrednosti in jih ponovno aktiviral).

\subsection{Pregled porabe električne energije:}
Uporabnik bo lahko za vsako merilno mesto pregledal porabo v obliki grafa. Podatke bo mogoče filtrirati po merilnih mestih. Mogoč bo tudi prikaz večjih merilnih mest skupaj. Meritve bodo prikazane v trenutni porabi za določen čas ali pa v skupni porabi.

\subsection{Pregled porabe "v živo":}

\subsection{Pregled napak:}

\subsection{Pregled navodil:}

\subsection{Nastavitev stikal za vklop/izklop merilnih mest/porabnikov:}

\chapter{Načrtovanje sistema}
\thispagestyle{fancy}
\pagenumbering{arabic}

Sistem bo zgrajen iz dveh enot. 
\begin{itemize}
\item Merilni sistem
\item Aplikacija
\end{itemize}

\section{Načrtovanje merilnega sistema - Komponent}

\subsection{Arduino}
Arduino je odprtokodna elektronska platforma zgrajena z hardwerom in softwerom ki je lahek za uporabo. Nastal je na inštitutu Ivrea Interaction Design, kot orodje za hitro izdelavo prototipov, namenjen pa je bil za študente brez predhodnega znanja elektrotehnike in programiranja. Kmalu za tem je postal zelo priljubljen in posledično se je začel razvijat. Zmore od enostavnih projektov kot so aktivacija motorja, prižiganje luči, branje senzorja pa do 3D printanja in uporabo v vgrajenih sistemih, uporabljajo ga pa tako začetniki kot tudi profesionalci. Na voljo je več verzij Arduinota, ki pa se razlikujejo predvsem v verziji mikrokontrolerja od katerega je tudi odvisno številov digitalnih in analognih vhodov in izhodov ter ceni, ki znaša od od 20 do 80 evrov. Ker je platforma kot že omenjeno odprtokodna pa seveda lahko dobimo Arduino ploščice pod drugim imenom, z enako funkcionalnostjo za veliko ceneje. Zaradi cene, razširjene uporabe platforme in obsežne dokumentacije, ki je na voljo na spletu, je bil Arduino kandidat za izdelavo naprave za diplomsko nalogo. Kasneje se je izkazalo, da je Raspberry pi nekoliko bolj     primeren.

\subsection{Raspberry Pi}
Kot že omenjeni Arduino je bil Raspberry Pi namenjen promoviranju in učenju osnov računalništva. Vsi modeli Raspberry Pi vsebujejo Broadcom-ov Soc(System on a chip), kateri je zgrajen iz ARM-ovega procesorja (CPU) in grafičnega procesorja GPU. Ure procesorjev so med 700MHz in 1.2GHz, pomnilnik pa varira med 256MB in 1GB RAM. Za shrambo operacijskega sistema je uporabljena Secure Digital (SD) kartica različnih velikosti. Večina ploščic ima med 1 in 4 USB vhodov, HDMI in composit video izhod, 3.5mm audio jack. Nižje razredni vhodi/izhodi so na voljo preko številnih GPIO pinov, ki podpirajo protokole kot na primer I2C. Nekatere ploščice imajo tudi Ethernet port in WIFI ter bluetooth. Deluje na operacijskem sistemu raspbian, ki je Debianova različica linuxove distribucije.

\subsection{Arduino ali Raspberry Pi}
Za razvoj merilnika za naše potrebe, razvojna ploščica Arduino zadostuje vsem zahtevam, zato je bila prva verzija merilnega instrumenta izdelana z Arduino ploščico verzijo Arduino UNO. Arduino je s pomočjo senzorjev uspešno opravljal meritve, vendar je bil nestabilen in sicer pri večkratnem zaporednem pošiljanju requestov na strežnik  je "zmrznil". Ker težave nismo uspeli odpraviti smo zato Arduino zamenjali z Raspberry Pi. Raspberry Pi se je izkazal enostavnejši za uporabo ( naprednejši jezik???? ), vendar smo pri zamenjavi naleteli na novo težavo in sicer Raspberry Pi nima analognih vhodov ( Arduino ima vgrajene analogno digitalne pretvornike na ploščici). Problem smo rešili z uporabo analogno/digitalnega pretvornika MCP3008.

\subsection{Analogno digitalni pretvornik mcp3008}
MCP3008 je nizkocenovni 8 kanalni 10 bitni analogno digitalni pretvornik. Natančnost tega čipa je podobna čipu uporabljenem na Arduino UNO. V primeru da bi kasneje želeli natančnejše meritve bi morali ta čip zamenjati za natančnejši čip na primer: ADS1015(12 bit) ali še natančnejši ADS1115(16 bit). Komunikacija med Raspberry Pi in MCP3008 poteka po SPI protokolu.





\section{Načrtovanje aplikacije}

\chapter{Izvedba}
\thispagestyle{fancy}
\pagenumbering{arabic}

\chapter{Testiranje}
\thispagestyle{fancy}
\pagenumbering{arabic}

\chapter{Zaključek}
\thispagestyle{fancy}
\pagenumbering{arabic}





\medskip
\noindent Zgled citiranja:

\medskip
Minimiziranje pasovnosti matrik pomaga pri njihovem shranjevanju in pri računanju z njimi, npr.~pri Gaussovi eliminaciji. Bralec bo podrobnosti našel v~\cite{Chinn, George, Strang}.


\chapter{Vložitve}
\thispagestyle{fancy}

Dodamo vezno besedilo.

\section{Široke vložitve}\label{siroke}

Dodamo vezno besedilo.

\subsection{Podpoglavje poglavja Široke vložitve}

Dodamo vezno besedilo.

\begin{defi}
Graf $G$ je {\em povezan}, če za vsaki dve točki $u,v\in V(G)$ obstaja vsaj ena $u$-$v$ pot v $G$.
\end{defi}

\begin{lema}
Lema je pomožna trditev, ki služi za dokaz glavnega izreka.
\end{lema}

\begin{proof}
Tu napišimo dokaz leme. Dokaz naj bo čim krajši, vendar razumljiv vsem študentom. Pazite na logično strukturo dokaza: Vsi koraki naj bodo utemeljeni.
\end{proof}

\begin{izr}
Izrek je najpomemnejša trditev v poglavju. Izrekov naj bo čim manj, preostale trditve formuliramo kot leme ali kot trditve.
\end{izr}

\begin{proof}
Tu napišemo dokaz izreka.
\end{proof}

\begin{posl}
Posledica je ugotovitev, ki neposredno sledi iz glavnega izreka. Potrebuje le krajši dokaz (par vrstic). Če se ne da dokazati v par vrsticah, potem to ni več posledica, temveč lema ali trditev.
\end{posl}

\begin{prim}
Z zgledom osvetlimo lemo ali glavni izrek. Zgled je lahko protiprimer k veljavnosti izreka, če mu izpustimo kakšno od hipotez.
\end{prim}

\newpage
Takole se vstavlja slika:

\bigskip
\bigskip
\begin{figure}[h!]
\begin{center}
\includegraphics[width=0.5\linewidth]{graf.pdf}
\end{center}
\caption{Vhodni podatki.}\label{slika:podatki}
\end{figure}

\newpage
 Takole se vstavlja tabela.

\begin{table}[h!]
\caption{Algoritem PLOGBAND}
\label{tabela:algoritem}
\fbox{%
      \parbox{\linewidth}{%
{\noindent \bf Algoritem PLOGBAND:}
\vskip 5pt
\indent Podatka: {\obeylines \indent \indent graf $G = G(V,E)$ na $n$ vozliščih in z $m$ povezavami,
% \indent \indent nenegativno celo število $L$.}
\indent \indent $L \in \N$.}
\begin{enumerate}

\item Za $1 \le j \le L$ naj bodo $p_j$ približno enakomerno
(geometrijsko) razporejena števila med $1 - 1/\log\log n$ in $1/\log n$.
Tj., vsa razmerja $p_j/p_{j+1}$ naj bodo približno enaka. (Natančne
formule so v~razdelku \ref{siroke}, kjer je opisana vložitev naključnih podmnožic.)

\item Uredi vozlišča glede na naraščajoče vrednosti $h(v)$. Vozlišča z enakimi
vrednostmi $h$ uredi poljubno.

\item Vrni urejeni seznam vozlišč kot linearno ureditev.
\end{enumerate}
}}
\end{table}

\noindent Takole navedemo sliko ali tabelo:

\medskip
Vse, kar potrebujemo za konec dokaza, je povzeto v Tabeli~\ref{tabela:algoritem}.

\medskip
Za primer vhodih podatkov glej Sliko~\ref{slika:podatki}.

\medskip
\noindent Podobno lahko označimo in navajamo razdelke, poglavja, izreke, ipd.


\newpage
Takole lahko zapišemo psevdokodo algoritma:

\begin{algorithm}[h!]\label{algoritem1}
\Vhod{Realni matriki $A$ in $B$ velikosti $n\times n$.}
\Izhod{Matrika $C = A\cdot B$.}
\caption{Množenje matrik}
{
    \Za{$i = 1, \ldots, n$}
    {
        \Za{$j = 1, \ldots, n$}
        {
            $C[i,j]:= A[i,1]\cdot B[1,j];$

            \Za{$k = 2, \ldots, n$}
            {
                $C[i,j]:= C[i,j]+A[i,k]\cdot B[k,j]$;
            }
        }
    }
    \Ce{$n = 2$}
    {
       ne naredi nič
    }
}
\Vrni{$C$;}
\end{algorithm}


\chapter{Naslov poglavja}
\thispagestyle{fancy}

Takole citiramo spletne vire:~\cite{splet1,splet2,splet3}.\\
Takole citiramo članke, sprejete v objavo ali v tisku:~\cite{Novak,Novak2,Novak3,Novak4}.\\
Takole citiramo članke, poslane v objavo:~\cite{Novak5,Novak6}.

%%%%%%%%%%%%%%%%%%%%%%%%%%%%%%%%%% Zaključek %%%%%%%%%%%%%%%%%%%%%%%%%%%%%%%%%%%%%
\chapter{Zaključek}
\thispagestyle{fancy}

V nekaj stavkih na kratko povzamemo, kaj smo v nalogi obravnavali.
Po želji lahko navedemo še kakšne dodatne reference za bralca, ki bi ga zanimalo kaj več, ipd.


%%%%%%%%%%%%%%%%%%%%%%%%%%%%%%%% Literatura %%%%%%%%%%%%%%%%%%%%%%%%%%%%%%%%%

 \begin{thebibliography}{99}
\thispagestyle{fancy}

\bibitem{Blum}
  \clanekVRevijiVecAvtorjev
    {A.~Blum, G.~Konjevod}{R.~Ravi}
    {Semidefinite relaxations for minimum bandwidth and other vertex-ordering problems}
   {Theor.~Comp.~Sci.}{235}
   {2000}{25--42}

\bibitem{Bourgain}
  \clanekVRevijiEnAvtor
    {J.~Bourgain}
    {On Lipschitz embedding of finite metric spaces in Hilbert space}
   {Israel J.~Math}{52}
   {1985}{46--52}

\bibitem{Chinn}
  \clanekVRevijiVecAvtorjev
    {P.~Chinn, J.~Chv\'atalov\'a, A.~Dewdney}{N.~Gibbs}
    {The bandwidth problem for graphs and matrices -- a survey}
   {J.~Graph Theory}{6}
   {1982}{223--254}

\bibitem{Chvatalova}
\doktorskaDisertacija
    {J.~Chv\'atalov\'a}
    {On the bandwidth problem for graphs}
    {Ph.D.~dissertation, University of Waterloo, 1980}

\bibitem{Frankl}
  \clanekVRevijiVecAvtorjev
    {P.~Frankl}{H.~Maehara}
    {The Johnson-Lindenstrauss lemma and the sphericity of some graphs}
   {J.~Comb.~Theory, Ser.~B}{44}
   {1988}{355--362}

\bibitem{Feige}
  \clanekVRevijiEnAvtor
    {U.~Feige}
    {Approximating the bandwidth via volume respecting embeddings}
    {J.~Comp.~Syst.~Sci.}{60}
    {2000}{510--539}

\bibitem{George}
  \knjigaVecAvtorjev {A.~George}{J.~Liu}
   {Computer Solution of Large Positive Definite Systems}
    {Prentice-Hall, 1981}

\bibitem{Grotschel}
  \knjigaVecAvtorjev  {M.~Gr\"otschel, L.~Lov\'asz}{A.~Schrijver}
   {Geometric Algorithms and Combinatorial Optimization}
    {Springer-Verlag, Berlin, 1987}

\bibitem{Kleitman}
   \clanekVRevijiVecAvtorjev
     {D.~Kleitman}{R.~Vohra}
     {Computing the bandwidth of interval graphs}
     {SIAM J.~Discrete Math.}{3}
     {1990}{373--375}

\bibitem{Knuth}
  \knjigaEnAvtor     {D.~Knuth}
   {The Art of Computer Programming, Vol.~2, Seminumerical Algorithms}
    {Addison Wesley, Second Edition, 1981}

\bibitem{Lagarias}
\poglavjeVKnjigiEnAvtor
   {J.~Lagarias}
   {Point Lattices}
   {R.~Graham, M.~Gr\"otschel, L.~Lov\'asz (ur.)}
   {Handbook of Combinatorics, Volume~1}
   {MIT Press, 1995}
   {919--966}

\bibitem{Linial}
   \clanekVRevijiVecAvtorjev
     {N.~Linial, E.~London}{Y.~Rabinovich}
   {The geometry of graphs and some of its algorithmic applications}
     {Combinatorica}{15}
     {1995}{215--245}

\bibitem{Novak}
   \clanekVRevijiEnAvtorSprejetVObjavo
     {J.~Novak}
   {Polynomial approximation of rational manifolds. I}
     {J.~Abstract~Approximation}

\bibitem{Novak2}
   \clanekVRevijiEnAvtorVTisku
     {J.~Novak}
   {Polynomial approximation of rational manifolds. II}
     {J.~Abstract~Approximation}

\bibitem{Novak3}
   \clanekVRevijiVecAvtorjevSprejetVObjavo
     {J.~Novak}{M.~Novak}
   {Polynomial approximation of rational manifolds. III}
     {J.~Abstract~Approximation}

\bibitem{Novak4}
   \clanekVRevijiVecAvtorjevVTisku
     {J.~Novak}{M.~Novak}
   {Polynomial approximation of rational manifolds. IV}
     {J.~Abstract~Approximation}

\bibitem{Novak5}
   \clanekVRevijiEnAvtorPoslanVObjavo
     {J.~Novak}
   {Polynomial approximation of rational manifolds. V}
     {2014}

\bibitem{Novak6}
   \clanekVRevijiVecAvtorjevPoslanVObjavo
     {J.~Novak}{M.~Novak}
   {Polynomial approximation of rational manifolds. VI}
     {2014}

\bibitem{Santalo}
  \knjigaEnAvtor     {L.A.~Santalo}
   {Integral Geometry and Geometric Probability}
    {Encyclopedia of Mathematics and its Applications, Volume 1, Addison Wesley, 1976}

\bibitem{Saxe}
   \clanekVRevijiEnAvtor
     {J.~Saxe}
   {Dynamic programming algorithms for recognizing small-bandwidth graphs in polynomial time}
     {SIAM J.~Alg.~Meth.}{1}
     {1980}{363--369}

\bibitem{Strang}
  \knjigaEnAvtor     {G.~Strang}
   {Linear Algebra and its Applications, Third Edition}
    {Saunders College \hbox{Publishing}, Harcourt Brace Jovanovich College Publishers, 1988}

\bibitem{Unger}
\konferencniClanekEnAvtor
    {W.~Unger}
    {The complexity of the approximation of the bandwidth problem}
    {Proc.~39th Annual IEEE Symposium on Foundations of Computer Science}
    {1998}
    {82--91}

\bibitem{splet1}
\spletniVirBrezAvtorja
    {Miller--Rabin primality test}
    {\newline http://en.wikipedia.org/wiki/Miller/\%E2\%80\%93Rabin\_primality\_test}
    {25}{4}{2014}

\bibitem{splet2}
\spletniVirBrezAvtorjaZInstitucijo
    {The Converse of Wilson's Theorem}{The Oxford Math Center}
    {http://www.oxfordmathcenter.com/drupal7/node/382}
    {25}{4}{2014}

\bibitem{splet3}
\spletniVirZAvtorjem
    {T.~Tao}
    {Algebraic probability spaces}
    {http://terrytao.wordpress.com/}
    {4}{7}{2014}

% Ena vrstica mora biti tu prazna zaradi pravilnih navedb na strani, kjer so reference citirane.
\end{thebibliography}
\newpage

%%%%%%%%%%%%%%%%%%%%%%%%%%%%%%%%%%%% Priloge %%%%%%%%%%%%%%%%%%%%%%%%%%%%%%%%%%%%%
\pagestyle{fancyplain}
\vspace*{\fill}
     \begin{center}
          \bf{\Huge{Priloge}}
     \end{center}
\vspace*{\fill}
\thispagestyle{fancy}

\appendix
\thispagestyle{empty}
\pagenumbering{gobble}

\addtocontents{toc}{\setcounter{tocdepth}{-1}}
\appendices{A Naslov prve priloge}
\chapter{Naslov prve priloge}
\thispagestyle{empty}
Tu dodamo prvo prilogo.

% pozor:
% ukaz
% \thispagestyle{empty}
% mora biti prisoten na vsaki strani priloge (da se ne prikaže glava dokumenta)

\appendices{B Naslov druge priloge}
\chapter{Naslov druge priloge}
\thispagestyle{empty}
Tu dodamo drugo prilogo.

% Pozor:
% ukaz
% \thispagestyle{empty}
% mora biti prisoten na vsaki strani priloge (da se ne prikaže glava dokumenta)

\addtocontents{toc}{\setcounter{tocdepth}{2}}
\end{document}
